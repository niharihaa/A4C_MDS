% Options for packages loaded elsewhere
\PassOptionsToPackage{unicode}{hyperref}
\PassOptionsToPackage{hyphens}{url}
%
\documentclass[
]{article}
\usepackage{amsmath,amssymb}
\usepackage{iftex}
\ifPDFTeX
  \usepackage[T1]{fontenc}
  \usepackage[utf8]{inputenc}
  \usepackage{textcomp} % provide euro and other symbols
\else % if luatex or xetex
  \usepackage{unicode-math} % this also loads fontspec
  \defaultfontfeatures{Scale=MatchLowercase}
  \defaultfontfeatures[\rmfamily]{Ligatures=TeX,Scale=1}
\fi
\usepackage{lmodern}
\ifPDFTeX\else
  % xetex/luatex font selection
\fi
% Use upquote if available, for straight quotes in verbatim environments
\IfFileExists{upquote.sty}{\usepackage{upquote}}{}
\IfFileExists{microtype.sty}{% use microtype if available
  \usepackage[]{microtype}
  \UseMicrotypeSet[protrusion]{basicmath} % disable protrusion for tt fonts
}{}
\makeatletter
\@ifundefined{KOMAClassName}{% if non-KOMA class
  \IfFileExists{parskip.sty}{%
    \usepackage{parskip}
  }{% else
    \setlength{\parindent}{0pt}
    \setlength{\parskip}{6pt plus 2pt minus 1pt}}
}{% if KOMA class
  \KOMAoptions{parskip=half}}
\makeatother
\usepackage{xcolor}
\usepackage[margin=1in]{geometry}
\usepackage{color}
\usepackage{fancyvrb}
\newcommand{\VerbBar}{|}
\newcommand{\VERB}{\Verb[commandchars=\\\{\}]}
\DefineVerbatimEnvironment{Highlighting}{Verbatim}{commandchars=\\\{\}}
% Add ',fontsize=\small' for more characters per line
\usepackage{framed}
\definecolor{shadecolor}{RGB}{248,248,248}
\newenvironment{Shaded}{\begin{snugshade}}{\end{snugshade}}
\newcommand{\AlertTok}[1]{\textcolor[rgb]{0.94,0.16,0.16}{#1}}
\newcommand{\AnnotationTok}[1]{\textcolor[rgb]{0.56,0.35,0.01}{\textbf{\textit{#1}}}}
\newcommand{\AttributeTok}[1]{\textcolor[rgb]{0.13,0.29,0.53}{#1}}
\newcommand{\BaseNTok}[1]{\textcolor[rgb]{0.00,0.00,0.81}{#1}}
\newcommand{\BuiltInTok}[1]{#1}
\newcommand{\CharTok}[1]{\textcolor[rgb]{0.31,0.60,0.02}{#1}}
\newcommand{\CommentTok}[1]{\textcolor[rgb]{0.56,0.35,0.01}{\textit{#1}}}
\newcommand{\CommentVarTok}[1]{\textcolor[rgb]{0.56,0.35,0.01}{\textbf{\textit{#1}}}}
\newcommand{\ConstantTok}[1]{\textcolor[rgb]{0.56,0.35,0.01}{#1}}
\newcommand{\ControlFlowTok}[1]{\textcolor[rgb]{0.13,0.29,0.53}{\textbf{#1}}}
\newcommand{\DataTypeTok}[1]{\textcolor[rgb]{0.13,0.29,0.53}{#1}}
\newcommand{\DecValTok}[1]{\textcolor[rgb]{0.00,0.00,0.81}{#1}}
\newcommand{\DocumentationTok}[1]{\textcolor[rgb]{0.56,0.35,0.01}{\textbf{\textit{#1}}}}
\newcommand{\ErrorTok}[1]{\textcolor[rgb]{0.64,0.00,0.00}{\textbf{#1}}}
\newcommand{\ExtensionTok}[1]{#1}
\newcommand{\FloatTok}[1]{\textcolor[rgb]{0.00,0.00,0.81}{#1}}
\newcommand{\FunctionTok}[1]{\textcolor[rgb]{0.13,0.29,0.53}{\textbf{#1}}}
\newcommand{\ImportTok}[1]{#1}
\newcommand{\InformationTok}[1]{\textcolor[rgb]{0.56,0.35,0.01}{\textbf{\textit{#1}}}}
\newcommand{\KeywordTok}[1]{\textcolor[rgb]{0.13,0.29,0.53}{\textbf{#1}}}
\newcommand{\NormalTok}[1]{#1}
\newcommand{\OperatorTok}[1]{\textcolor[rgb]{0.81,0.36,0.00}{\textbf{#1}}}
\newcommand{\OtherTok}[1]{\textcolor[rgb]{0.56,0.35,0.01}{#1}}
\newcommand{\PreprocessorTok}[1]{\textcolor[rgb]{0.56,0.35,0.01}{\textit{#1}}}
\newcommand{\RegionMarkerTok}[1]{#1}
\newcommand{\SpecialCharTok}[1]{\textcolor[rgb]{0.81,0.36,0.00}{\textbf{#1}}}
\newcommand{\SpecialStringTok}[1]{\textcolor[rgb]{0.31,0.60,0.02}{#1}}
\newcommand{\StringTok}[1]{\textcolor[rgb]{0.31,0.60,0.02}{#1}}
\newcommand{\VariableTok}[1]{\textcolor[rgb]{0.00,0.00,0.00}{#1}}
\newcommand{\VerbatimStringTok}[1]{\textcolor[rgb]{0.31,0.60,0.02}{#1}}
\newcommand{\WarningTok}[1]{\textcolor[rgb]{0.56,0.35,0.01}{\textbf{\textit{#1}}}}
\usepackage{graphicx}
\makeatletter
\def\maxwidth{\ifdim\Gin@nat@width>\linewidth\linewidth\else\Gin@nat@width\fi}
\def\maxheight{\ifdim\Gin@nat@height>\textheight\textheight\else\Gin@nat@height\fi}
\makeatother
% Scale images if necessary, so that they will not overflow the page
% margins by default, and it is still possible to overwrite the defaults
% using explicit options in \includegraphics[width, height, ...]{}
\setkeys{Gin}{width=\maxwidth,height=\maxheight,keepaspectratio}
% Set default figure placement to htbp
\makeatletter
\def\fps@figure{htbp}
\makeatother
\setlength{\emergencystretch}{3em} % prevent overfull lines
\providecommand{\tightlist}{%
  \setlength{\itemsep}{0pt}\setlength{\parskip}{0pt}}
\setcounter{secnumdepth}{-\maxdimen} % remove section numbering
\ifLuaTeX
  \usepackage{selnolig}  % disable illegal ligatures
\fi
\usepackage{bookmark}
\IfFileExists{xurl.sty}{\usepackage{xurl}}{} % add URL line breaks if available
\urlstyle{same}
\hypersetup{
  pdftitle={Multidimensional-Scaling.R},
  pdfauthor={Nihariha Kamal},
  hidelinks,
  pdfcreator={LaTeX via pandoc}}

\title{Multidimensional-Scaling.R}
\author{Nihariha Kamal}
\date{2024-07-09}

\begin{document}
\maketitle

\begin{Shaded}
\begin{Highlighting}[]
\CommentTok{\# Install and load necessary libraries}
\ControlFlowTok{if}\NormalTok{ (}\SpecialCharTok{!}\FunctionTok{requireNamespace}\NormalTok{(}\StringTok{"pheatmap"}\NormalTok{, }\AttributeTok{quietly =} \ConstantTok{TRUE}\NormalTok{)) \{}
  \FunctionTok{install.packages}\NormalTok{(}\StringTok{"pheatmap"}\NormalTok{)}
\NormalTok{\}}
\FunctionTok{library}\NormalTok{(ggplot2)}
\FunctionTok{library}\NormalTok{(scales)}
\FunctionTok{library}\NormalTok{(dplyr)}
\end{Highlighting}
\end{Shaded}

\begin{verbatim}
## 
## Attaching package: 'dplyr'
\end{verbatim}

\begin{verbatim}
## The following objects are masked from 'package:stats':
## 
##     filter, lag
\end{verbatim}

\begin{verbatim}
## The following objects are masked from 'package:base':
## 
##     intersect, setdiff, setequal, union
\end{verbatim}

\begin{Shaded}
\begin{Highlighting}[]
\FunctionTok{library}\NormalTok{(tidyr)}
\FunctionTok{library}\NormalTok{(readr)}
\end{Highlighting}
\end{Shaded}

\begin{verbatim}
## 
## Attaching package: 'readr'
\end{verbatim}

\begin{verbatim}
## The following object is masked from 'package:scales':
## 
##     col_factor
\end{verbatim}

\begin{Shaded}
\begin{Highlighting}[]
\FunctionTok{library}\NormalTok{(pheatmap)}

\CommentTok{\# Load the dataset}
\NormalTok{data\_filepath }\OtherTok{\textless{}{-}} \StringTok{"C:/Users/nihar/OneDrive/Desktop/Bootcamp/SCMA 632/DataSet/icecream.csv"}
\NormalTok{icecream\_data }\OtherTok{\textless{}{-}} \FunctionTok{read\_csv}\NormalTok{(data\_filepath)}
\end{Highlighting}
\end{Shaded}

\begin{verbatim}
## Rows: 10 Columns: 7
\end{verbatim}

\begin{verbatim}
## -- Column specification --------------------------------------------------------
## Delimiter: ","
## chr (1): Brand
## dbl (6): Price, Availability, Taste, Flavour, Consistency, Shelflife
## 
## i Use `spec()` to retrieve the full column specification for this data.
## i Specify the column types or set `show_col_types = FALSE` to quiet this message.
\end{verbatim}

\begin{Shaded}
\begin{Highlighting}[]
\CommentTok{\# Display the first few rows of the dataset}
\FunctionTok{head}\NormalTok{(icecream\_data)}
\end{Highlighting}
\end{Shaded}

\begin{verbatim}
## # A tibble: 6 x 7
##   Brand   Price Availability Taste Flavour Consistency Shelflife
##   <chr>   <dbl>        <dbl> <dbl>   <dbl>       <dbl>     <dbl>
## 1 Amul        4            5     4       3           4         3
## 2 Nandini     3            2     3       2           3         3
## 3 Vadilal     2            2     4       3           4         4
## 4 Vijaya      3            1     3       5           3         4
## 5 Dodla       3            3     3       4           4         3
## 6 Hatson      2            2     4       4           3         4
\end{verbatim}

\begin{Shaded}
\begin{Highlighting}[]
\CommentTok{\# Check the structure of the dataset}
\FunctionTok{str}\NormalTok{(icecream\_data)}
\end{Highlighting}
\end{Shaded}

\begin{verbatim}
## spc_tbl_ [10 x 7] (S3: spec_tbl_df/tbl_df/tbl/data.frame)
##  $ Brand       : chr [1:10] "Amul" "Nandini" "Vadilal" "Vijaya" ...
##  $ Price       : num [1:10] 4 3 2 3 3 2 2 4 3 4
##  $ Availability: num [1:10] 5 2 2 1 3 2 3 1 4 2
##  $ Taste       : num [1:10] 4 3 4 3 3 4 4 2 5 3
##  $ Flavour     : num [1:10] 3 2 3 5 4 4 3 3 5 2
##  $ Consistency : num [1:10] 4 3 4 3 4 3 4 3 4 3
##  $ Shelflife   : num [1:10] 3 3 4 4 3 4 4 3 4 3
##  - attr(*, "spec")=
##   .. cols(
##   ..   Brand = col_character(),
##   ..   Price = col_double(),
##   ..   Availability = col_double(),
##   ..   Taste = col_double(),
##   ..   Flavour = col_double(),
##   ..   Consistency = col_double(),
##   ..   Shelflife = col_double()
##   .. )
##  - attr(*, "problems")=<externalptr>
\end{verbatim}

\begin{Shaded}
\begin{Highlighting}[]
\CommentTok{\# Select only the numeric columns for MDS}
\NormalTok{icecream\_data\_numeric }\OtherTok{\textless{}{-}}\NormalTok{ icecream\_data }\SpecialCharTok{\%\textgreater{}\%}
  \FunctionTok{select}\NormalTok{(}\SpecialCharTok{{-}}\NormalTok{Brand)}

\CommentTok{\# Verify the cleaned data}
\FunctionTok{str}\NormalTok{(icecream\_data\_numeric)}
\end{Highlighting}
\end{Shaded}

\begin{verbatim}
## tibble [10 x 6] (S3: tbl_df/tbl/data.frame)
##  $ Price       : num [1:10] 4 3 2 3 3 2 2 4 3 4
##  $ Availability: num [1:10] 5 2 2 1 3 2 3 1 4 2
##  $ Taste       : num [1:10] 4 3 4 3 3 4 4 2 5 3
##  $ Flavour     : num [1:10] 3 2 3 5 4 4 3 3 5 2
##  $ Consistency : num [1:10] 4 3 4 3 4 3 4 3 4 3
##  $ Shelflife   : num [1:10] 3 3 4 4 3 4 4 3 4 3
\end{verbatim}

\begin{Shaded}
\begin{Highlighting}[]
\FunctionTok{summary}\NormalTok{(icecream\_data\_numeric)}
\end{Highlighting}
\end{Shaded}

\begin{verbatim}
##      Price       Availability     Taste        Flavour     Consistency 
##  Min.   :2.00   Min.   :1.0   Min.   :2.0   Min.   :2.0   Min.   :3.0  
##  1st Qu.:2.25   1st Qu.:2.0   1st Qu.:3.0   1st Qu.:3.0   1st Qu.:3.0  
##  Median :3.00   Median :2.0   Median :3.5   Median :3.0   Median :3.5  
##  Mean   :3.00   Mean   :2.5   Mean   :3.5   Mean   :3.4   Mean   :3.5  
##  3rd Qu.:3.75   3rd Qu.:3.0   3rd Qu.:4.0   3rd Qu.:4.0   3rd Qu.:4.0  
##  Max.   :4.00   Max.   :5.0   Max.   :5.0   Max.   :5.0   Max.   :4.0  
##    Shelflife  
##  Min.   :3.0  
##  1st Qu.:3.0  
##  Median :3.5  
##  Mean   :3.5  
##  3rd Qu.:4.0  
##  Max.   :4.0
\end{verbatim}

\begin{Shaded}
\begin{Highlighting}[]
\CommentTok{\# Compute the distance matrix}
\NormalTok{icecream\_dist }\OtherTok{\textless{}{-}} \FunctionTok{dist}\NormalTok{(icecream\_data\_numeric)}

\CommentTok{\# Apply Multidimensional Scaling (MDS)}
\NormalTok{mds\_fit }\OtherTok{\textless{}{-}} \FunctionTok{cmdscale}\NormalTok{(icecream\_dist, }\AttributeTok{k =} \DecValTok{2}\NormalTok{)  }\CommentTok{\# k = 2 for 2D plot}

\CommentTok{\# Create a data frame with MDS results}
\NormalTok{mds\_data }\OtherTok{\textless{}{-}} \FunctionTok{as.data.frame}\NormalTok{(mds\_fit)}
\FunctionTok{colnames}\NormalTok{(mds\_data) }\OtherTok{\textless{}{-}} \FunctionTok{c}\NormalTok{(}\StringTok{"Dim1"}\NormalTok{, }\StringTok{"Dim2"}\NormalTok{)}
\NormalTok{mds\_data}\SpecialCharTok{$}\NormalTok{Sample }\OtherTok{\textless{}{-}}\NormalTok{ icecream\_data}\SpecialCharTok{$}\NormalTok{Brand}

\CommentTok{\# Plot the MDS results}
\NormalTok{mds\_plot }\OtherTok{\textless{}{-}} \FunctionTok{ggplot}\NormalTok{(mds\_data, }\FunctionTok{aes}\NormalTok{(}\AttributeTok{x =}\NormalTok{ Dim1, }\AttributeTok{y =}\NormalTok{ Dim2, }\AttributeTok{label =}\NormalTok{ Sample)) }\SpecialCharTok{+}
  \FunctionTok{geom\_point}\NormalTok{(}\AttributeTok{size =} \DecValTok{3}\NormalTok{) }\SpecialCharTok{+}
  \FunctionTok{geom\_text}\NormalTok{(}\AttributeTok{vjust =} \SpecialCharTok{{-}}\FloatTok{0.5}\NormalTok{) }\SpecialCharTok{+}
  \FunctionTok{labs}\NormalTok{(}\AttributeTok{title =} \StringTok{"MDS Plot of Ice Cream Dataset"}\NormalTok{,}
       \AttributeTok{x =} \StringTok{"Dimension 1"}\NormalTok{,}
       \AttributeTok{y =} \StringTok{"Dimension 2"}\NormalTok{) }\SpecialCharTok{+}
  \FunctionTok{theme\_minimal}\NormalTok{()}

\CommentTok{\# Show the MDS plot}
\FunctionTok{print}\NormalTok{(mds\_plot)}
\end{Highlighting}
\end{Shaded}

\includegraphics{Multidimensional-Scaling_files/figure-latex/unnamed-chunk-1-1.pdf}

\begin{Shaded}
\begin{Highlighting}[]
\CommentTok{\# Create a heatmap of the distance matrix}
\NormalTok{heatmap\_data }\OtherTok{\textless{}{-}} \FunctionTok{as.matrix}\NormalTok{(icecream\_dist)}
\FunctionTok{rownames}\NormalTok{(heatmap\_data) }\OtherTok{\textless{}{-}}\NormalTok{ icecream\_data}\SpecialCharTok{$}\NormalTok{Brand}
\FunctionTok{colnames}\NormalTok{(heatmap\_data) }\OtherTok{\textless{}{-}}\NormalTok{ icecream\_data}\SpecialCharTok{$}\NormalTok{Brand}

\CommentTok{\# Plot the heatmap}
\NormalTok{heatmap\_plot }\OtherTok{\textless{}{-}} \FunctionTok{pheatmap}\NormalTok{(heatmap\_data,}
                         \AttributeTok{clustering\_distance\_rows =}\NormalTok{ icecream\_dist,}
                         \AttributeTok{clustering\_distance\_cols =}\NormalTok{ icecream\_dist,}
                         \AttributeTok{display\_numbers =} \ConstantTok{TRUE}\NormalTok{,}
                         \AttributeTok{fontsize\_number =} \DecValTok{8}\NormalTok{,}
                         \AttributeTok{main =} \StringTok{"Heatmap of Ice Cream Distance Matrix"}\NormalTok{)}
\end{Highlighting}
\end{Shaded}

\includegraphics{Multidimensional-Scaling_files/figure-latex/unnamed-chunk-1-2.pdf}

\end{document}
